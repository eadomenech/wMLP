% This is samplepaper.tex, a sample chapter demonstrating the
% LLNCS macro package for Springer Computer Science proceedings;
% Version 2.20 of 2017/10/04
%
\documentclass[runningheads]{llncs}
%
\usepackage{graphicx}
% Used for displaying a sample figure. If possible, figure files should
% be included in EPS format.
%
% If you use the hyperref package, please uncomment the following line
% to display URLs in blue roman font according to Springer's eBook style:
% \renewcommand\UrlFont{\color{blue}\rmfamily}

\begin{document}
%
\title{Fragile watermarking method for handwritten document image integrity and tamper detection}
%
%\titlerunning{Abbreviated paper title}
% If the paper title is too long for the running head, you can set
% an abbreviated paper title here
%
\author{Ernesto Avila-Domenech\inst{1}\orcidID{0000-0002-4797-289X} \and
Alberto Taboada-Crispi\inst{2}\orcidID{0000-0002-7797-1441} \and
Anier Soria-Lorente\inst{1}\orcidID{0000-0003-3488-3094}}
%
\authorrunning{E. Avila-Domenech et al.}
% First names are abbreviated in the running head.
% If there are more than two authors, 'et al.' is used.
%
\institute{Universidad de Granma, Carretera Central v{\'i}a Holgu{\'i}n Km $\frac{1}{2}$, Granma, Cuba \email{\{eadomenech, asorial1983\}@gmail.com}\\ \and
Universidad Central de Las Villas, Villa Clara, Cuba\\
\email{\{abc,lncs\}@uni-heidelberg.de}}
%
\maketitle              % typeset the header of the contribution
%
\begin{abstract}
In this paper, by using SHA-256 hash function, a fragile watermarking for handwritten document image integrity and tamper detection method is proposed.The watermark is constructed from the image to be watermarked.

\keywords{First keyword  \and Second keyword \and Another keyword.}
\end{abstract}
%
%
%
\section{Introduction}
Many types of watermarking techniques have been developed for a variety of applications.

To solve this problem, methods are developed that can be classified into digital signature-based methods and watermarking-based methods.

A digital signature is a set of features extracted from an image and these are stored in a separate file. Digital watermarking consists of inserting imperceptible information called watermark in a multimedia document to protect the copyright of the document while ensuring its effective transmission \cite{el2014image}.

Watermarks may be visible or invisible, where a visible mark is easily detected by observation while an invisible mark is designed to be transparent to the observer and detected using signal processing techniques.

A robust mark is designed to resist attacks that attempt to remove or destroy the mark. Such attacks include
lossy compression, filtering, and geometric scaling. A fragile mark is designed to detect slight changes to the watermarked image with high probability.

The main application of fragile watermarks is in content authentication.

Watermarks may be visible or invisible, where a visible mark is easily detected by observation while an invisible mark is designed to be transparent to the observer and detected using signal processing techniques.

A robust mark is designed to resist attacks that attempt to remove or destroy the mark. Such attacks include lossy compression, filtering, and geometric scaling. A fragile mark is designed to detect slight changes to the watermarked image with high probability.

Different with the robust image watermarking that is robust to a variety of attacks for ownership and copyright protection, fragile image watermarking is utilized for the integrity authentication, which is fragile to the illegal tampering.

The main application of fragile watermarks is in content authentication. That is, it may be of interest for parties to verify that an image has not been edited, damaged, or altered since it was marked.

While the purpose of fragile watermarking and digital signature systems are similar, watermarking systems offer several advantages. Since a watermark is embedded directly in the image data, no additional information is necessary for authenticity verification. This is unlike digital signatures since the signature itself must be bound to the transmitted data. Therefore the critical information needed in the authenticity testing process is discreetly hidden and more difficult to remove than a digital signature. Therefore a signature system may be able to detect that an image had been modified but cannot characterise the alterations. Many watermarking systems can determine which areas of a marked image have been altered and which areas have not.

Authentication and integrity systems of image can be grouped in several ways depending on the mode of storage
of authentication data that is based techniques on electronic signature based or the fragile watermarking or
even depending on the nature of the information they burrow into the document to protect. The main difference
between these two categories of techniques is that in the digital signature techniques, the authentication data is
transmitted in a separate of the raw data stored in the same folder. While in watermarking techniques, the
authentication data are embedded in the raw data. In the remainder of this paper, we present a technique developed
based on the fragile watermarking. \cite{boujemaa2016fragile}

Exist several desirable features of fragile watermarking methods bat the most important are:

\begin{enumerate}
	\item \textbf{Perceptual transparency}: An embedded watermark should not be visible under normal observation or interfere	with the functionality of the image.
	\item \textbf{Detect tampering}: A fragile marking system should detect with high probability any tampering in a marked image.
	\item \textbf{Detection should not require the original image}:
\end{enumerate}

In \cite{gul2019novel} host image is divided into $32\times 32$ non-overlapped blocks, each $32\times 32$ block is then divided into four $16\times 16$ nonoverlapped sub-blocks. The entire hash value of the first three sub-blocks is generated as a watermark using SHA-256 hash function. The generated 256-bit binary watermark is embedded into the least significant bits (LSBs) of the fourth sub-block and watermarked image is obtained.

In the watermarking techniques developed for medical applications, information generated from the region of interest (ROI) of medical image is generally embedded into the region of noninterest (RONI) of the image. However, in these
techniques, tamper detection can be performed inside ROI area.

The rest of the paper is organised as follow; Section 2 describes the proposed method including watermark insertion, watermark extraction, tamper localisation and reconstruction. Experimental results are given in Section 3 and Section 4 concludes the paper.

\section{Proposed method}
As we know, hash function, such as MD5, can be utilized to authenticate the data integrity. If the hash value of original message is exactly equal to the re-calculated hash value of the received message, the received data can be regarded as integrated, otherwise as false.

\subsection{Variants only for image integrity and tamper detection}
\subsubsection{Variants 1}
Generic variant.

Host image is divided into $16\times 16$ non-overlapped blocks, each $16\times 16$ block is then divided into four $8\times 8$ nonoverlapped sub-blocks. The complete block is pariado making use of a given key. The entire hash value of the block pariado is generated as a watermark using MD5 hash function. The generated MD5 binary watermark is embedded into the least significant bits (LSBs) of the pixels pariados and watermarked image is obtained.

\subsection{Variants for image integrity, tamper detection and self-recovering}
\subsubsection{Variants 1}
Variant for handwritten documents images.

\begin{itemize}
	\item Geometric layout analysis to classify in region of interest (ROI) or non-interest (RONI). 
	\item ROI is divided into $16\times 16$ non-overlapped blocks, each $16\times 16$ block is then divided into four $8\times 8$ nonoverlapped sub-blocks. The complete block is pariado making use of a given key. The entire hash value of the block pariado is generated as a watermark using MD5 hash function. The generated MD5 binary watermark is embedded into the least significant bits (LSBs) of the pixels pariados and watermarked image is obtained.
	\item The main characteristics of the blocks belonging to the ROI are embedded in the blocks corresponding to the RONI.
	\item Like \cite{el2014image}, the neural network is utilised to recover a tampered $16\times 16$ block of an image.  
\end{itemize}



\section{Experiments and Results}
Subsequent paragraphs, however, are indented.

\subsection{Invisibility results}
The imperceptibility of the watermark was evaluated using several databases. We calculated the PSNR which compares the similarity between the original image $ I $ and the watermarked image $ I_w $.

\subsection{Tamper detection}
Tamper area detection capability is evaluated, by modifying the contents of images, adding objects or deleting objects.

\subsection{Tamper reconstruction}
Fig. 13 shows reconstructed images by applying the proposed algorithm on tampered regions.

\paragraph{Sample Heading (Fourth Level)}
The contribution should contain no more than four levels of
headings. Table~\ref{tab1} gives a summary of all heading levels.

\begin{table}
\caption{Table captions should be placed above the
tables.}\label{tab1}
\begin{tabular}{|l|l|l|}
\hline
Heading level &  Example & Font size and style\\
\hline
Title (centered) &  {\Large\bfseries Lecture Notes} & 14 point, bold\\
1st-level heading &  {\large\bfseries 1 Introduction} & 12 point, bold\\
2nd-level heading & {\bfseries 2.1 Printing Area} & 10 point, bold\\
3rd-level heading & {\bfseries Run-in Heading in Bold.} Text follows & 10 point, bold\\
4th-level heading & {\itshape Lowest Level Heading.} Text follows & 10 point, italic\\
\hline
\end{tabular}
\end{table}


\noindent Displayed equations are centered and set on a separate
line.
\begin{equation}
x + y = z
\end{equation}
Please try to avoid rasterized images for line-art diagrams and
schemas. Whenever possible, use vector graphics instead (see
Fig.~\ref{fig1}).

\begin{figure}
\includegraphics[width=\textwidth]{fig1.eps}
\caption{A figure caption is always placed below the illustration.
Please note that short captions are centered, while long ones are
justified by the macro package automatically.} \label{fig1}
\end{figure}

\begin{theorem}
This is a sample theorem. The run-in heading is set in bold, while
the following text appears in italics. Definitions, lemmas,
propositions, and corollaries are styled the same way.
\end{theorem}
%
% the environments 'definition', 'lemma', 'proposition', 'corollary',
% 'remark', and 'example' are defined in the LLNCS documentclass as well.
%
\begin{proof}
Proofs, examples, and remarks have the initial word in italics,
while the following text appears in normal font.
\end{proof}
For citations of references, we prefer the use of square brackets
and consecutive numbers. Citations using labels or the author/year
convention are also acceptable. The following bibliography provides
a sample reference list with entries for journal
articles~\cite{ref_article1}, an LNCS chapter~\cite{ref_lncs1}, a
book~\cite{ref_book1}, proceedings without editors~\cite{ref_proc1},
and a homepage~\cite{ref_url1}. Multiple citations are grouped
\cite{ref_article1,ref_lncs1,ref_book1},
\cite{ref_article1,ref_book1,ref_proc1,ref_url1}.
\section{Conclusions}
Conclusion
%
% ---- Bibliography ----
%
% BibTeX users should specify bibliography style 'splncs04'.
% References will then be sorted and formatted in the correct style.
%
\bibliographystyle{splncs04}
\bibliography{mybibliography}
%
\end{document}
